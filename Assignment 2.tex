\documentclass[11pt,a4paper,oneside]{article}
\usepackage[latin1]{inputenc}
\usepackage{amsmath}
\usepackage{amsfonts}
\usepackage{amssymb}
\usepackage{graphicx}
\usepackage{color}
\usepackage {tikz}
\usetikzlibrary {er}
\usepackage{multicol}
\newcommand \tab[1][1cm]{\hspace*{#1}}
\setlength{\columnseprule}{0.4pt}
\usepackage[left=2.00cm, right=2.00cm, top=1.00cm]{geometry}

\begin{document}
	\title{DS 221 - Introduction to Scalable Systems \\ Homework 2 Solutions}
	\author{Shriram R. \\ M Tech (CDS) \\ 06-02-01-10-51-18-1-15763}
	\maketitle
	\begin{enumerate}{}{}
		\item (2.82) A) The formula can be derived by shifting the value by $k$ bits as follows, \\
		      \begin{equation}
		      K = 0.yyy...
		      \end{equation}
		      \begin{equation}
	          2^k K = y.yyy...
	          \end{equation}
	          \tab Subtracting (1) from (2) and rearranging terms we get, \\
	          \begin{equation}
	          K = \frac{y}{2^k-1}
	          \end{equation}
	          \newline
	         (2.82) B) The numeric value of the given strings are: \\
	         \tab a) 101 --- $ y = 101, k=3, K = \frac{5}{7}$ (5 by 7) \\
	         \tab b) 0110 --- $y = 0110, k = 4, K = \frac{6}{15}$ (6 by 15)\\
	         \tab c) 010011 --- $y = 010011, k = 6, K = \frac{19}{63}$ (19 by 63)\\           
		\begin{multicols}{2}
		\item (2.89) The C program is given below,
		\begin{verbatim}
	    float fpwr2(int x)
	    {
	      unsigned exp, frac;
	      unsigned u;
	      
	      if (x < -149) {
	         exp = 0;
	         frac = 0;
	      } else if (x < -126) {
	         exp = 0;
	         frac = 1 << (149+x);
	      } else if (x < 128) {
	         exp = x + 127;
	         frac = 0;
	      } else {
	         exp = 255;
	         frac = 0;
	      }
	      u = exp << 23 | frac;
	      return u2f(u);
	    }
		\end{verbatim}
		\columnbreak
		\item (2.90) A) The given 0x40490FDB value can be represented in binary as a IEEE 32 bit floating point value as follows, \\
		      \tab \tab $ s = 0 $ \\
		      \tab \tab $ e = 10000000_2 = 1\textsubscript{2} $ (w/o bias) \\
		      \tab \tab $ f = 10010010000111111011011_2 $ \\
		      Hence, binary value = $ 11.0010010000111111011011_2 $ \\ \\ \\
		      (2.90) B) The number $\frac{22}{7}$ can be represented in binary as follows, \\
		      \tab \tab $ \frac{22}{7} = 3\frac{1}{7} = 11_2 + U2B(\frac{1}{7})$ \\
		      Using the formula derived in (2.82), we see that $y=001$ and $k=3$ for $\frac{1}{7}$. Therefore, \\
		      \tab \tab $ \frac{22}{7} = 11.001001001..._2 $ \\ \\ \\
		      (2.90) C) The first and second approximation diverges from the $9$th bit position to the right of binary point.
		       
		
	\end{multicols}		
	\end{enumerate}
	
\end{document}