\documentclass[11pt,a4paper,oneside]{article}
\usepackage[latin1]{inputenc}
\usepackage{amsmath}
\usepackage{amsfonts}
\usepackage{amssymb}
\usepackage{graphicx}
\usepackage{color}
\usepackage {tikz}
\usetikzlibrary {er}
\usepackage[left=2.00cm, right=2.00cm, top=2.00cm]{geometry}

\begin{document}
	\title{DS 221 - Introduction to Scalable Systems \\ Assignment 1}
	\author{Shriram R. \\ M Tech (CDS) \\ 06-02-01-10-51-18-1-15763}
	\maketitle
	
	\section{Introduction}
	An efficient C program to multiply two square matrices has been developed by leveraging vectorization and locality of reference. The program has been tested for different matrix sizes and a comparison has been made with a naive implementation. The following sections will cover the program, analysis and performance results in detail.
	
	\section{Program}
	\begin{verbatim}
	#include <stdlib.h>
	
	double a[SIZE][SIZE], b[SIZE][SIZE], c[SIZE][SIZE]; // SIZE is set during compile
	
	int main()
	{
	    for(int i=0; i<SIZE; i++) {
	        for(int j=0; j<SIZE; j++) {
	            a[i][j]=rand(); // Initialize with a random integer
	            b[i][j]=rand(); // Initialize with a random integer
	            c[i][j]=0;      // Initialize result matrix with 0
	        }
	    }
	
	    for(int i=0; i<SIZE; i++) {
	        for(int j=0; j<SIZE; j++) {
            for(int k=0; k<SIZE; k++) {
                c[i][k] += a[i][j]*b[j][k];
            }
        }
    }
        	
	    return 0;
	}
	\end{verbatim}
	
    \section{Analysis}
    The key idea behind the program is that each row in the result matrix C can be interpretated as a linear combination of the rows in matrix B using the values in rows of matrix A as weights.
    
    \section{Performance}
    The program has been tested in a personal computer having Intel Core i5 processor with 6MB cache and 4GB RAM.

\end{document}