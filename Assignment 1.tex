\documentclass[11pt,a4paper,oneside]{article}
\usepackage[latin1]{inputenc}
\usepackage{amsmath}
\usepackage{amsfonts}
\usepackage{amssymb}
\usepackage{graphicx}
\usepackage{color}
\usepackage {tikz}
\usetikzlibrary {er}
\usepackage[left=2.00cm, right=2.00cm, top=1.00cm]{geometry}

\begin{document}
	\title{DS 221 - Introduction to Scalable Systems \\ Homework 1 Solutions}
	\author{Shriram R. \\ M Tech (CDS) \\ 06-02-01-10-51-18-1-15763}
	\maketitle
	\begin{enumerate}{}{}
		\item The file sizes of a.out for the Hello World program along with instruction count for different compiler optimization flags are tabulated below. gcc v7.3 on x86-64 linux was used for compilation. Instruction count was obtained using the objdump tool.
		\begin{center}
			\begin{tabular}{|l|l|l|l|}
				\hline
				Optimization & Description & File Size &  Instruction Count\\
				\hline
				O0 & Default & 8296 B & 292 \\
				O1 & Level 1 & 8296 B & 286 \\
				O2 & Level 2 & 8296 B & 286 \\
				O3 & Level 3 & 8296 B & 286 \\				
				\hline
			\end{tabular}
		\end{center}
		\item (2.62) The C program is given below,
		\begin{verbatim}
		int shifts_are_arithmetic() {
		 int a = -1; // Initialize with all bits set to 1		 
		 if(a >> 1 == -1) // For arithmetic right shift, sign bit remains constant
		  return 1;
		 else
		  return 0;
		}
		\end{verbatim}
		\item (2.68) The C program is given below,
		\begin{verbatim}
		int lower_one_mask(int n) {
		 int a = -1; // Initialize with all bits set to 1		 
		 // Shift happens in two steps to handle the case n = word size.
		 a = a << n-1;
		 a = a << 1;		 
		 return a ^ -1; // Complement by xoring with -1
		}
		\end{verbatim}	
		\item (2.70) The C program is given below,
		\begin{verbatim}
			int fits_bits(int x, int n) {
			 x = x >> n-1; // Arithmetic right shift is assumed here			 
			 // x should be 0 (all bits set to 0) or
			 // x should be -1 (all bits set to 1)			 
			 return x == 0 || x == -1;		
			}
		\end{verbatim}	
	\end{enumerate}
	
\end{document}